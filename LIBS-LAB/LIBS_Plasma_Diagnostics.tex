\documentclass[12pt,a4paper]{article}
\usepackage[utf8]{inputenc}
\usepackage[T1]{fontenc}
\usepackage{amsmath}
\usepackage{amsfonts}
\usepackage{amssymb}
\usepackage{graphicx}
\usepackage{booktabs}
\usepackage{array}
\usepackage{geometry}
\usepackage{fancyhdr}
\usepackage{hyperref}
\usepackage{xcolor}
\usepackage{listings}

\geometry{left=2.5cm,right=2.5cm,top=2.5cm,bottom=2.5cm}

\title{\textbf{LIBS等离子体诊断:温度和电子密度计算方法}}
\author{激光诱导击穿光谱技术文档}
\date{\today}

\begin{document}

\maketitle
\section{引言}

激光诱导击穿光谱(LIBS)技术中,等离子体的温度和电子密度是两个关键的物理参数。这些参数不仅决定了等离子体的物理化学性质,还直接影响光谱线的强度、展宽和形状,进而影响定量分析的准确性。本文档详细介绍了LIBS中等离子体温度和电子密度的主要计算方法。

\section{等离子体温度计算}

\subsection{Boltzmann图法}

\subsubsection{理论基础}

在局部热力学平衡(LTE)条件下,原子在不同能级上的布居遵循Boltzmann分布:

\begin{equation}
\frac{N_i}{N_0} = \frac{g_i}{g_0} \exp\left(-\frac{E_i}{kT}\right)
\end{equation}

其中:
\begin{itemize}
    \item $N_i$:第$i$能级的原子数密度
    \item $N_0$:基态原子数密度
    \item $g_i$:第$i$能级的统计权重
    \item $E_i$:第$i$能级的能量
    \item $k$:Boltzmann常数
    \item $T$:等离子体温度
\end{itemize}

\subsubsection{谱线强度关系}

谱线强度与能级布居的关系为:

\begin{equation}
I_{ij} = \frac{hc}{4\pi\lambda_{ij}} A_{ij} N_i
\end{equation}

其中:
\begin{itemize}
    \item $I_{ij}$:从能级$i$到能级$j$的谱线强度
    \item $h$:Planck常数
    \item $c$:光速
    \item $\lambda_{ij}$:跃迁波长
    \item $A_{ij}$:Einstein自发发射系数
\end{itemize}

\subsubsection{Boltzmann图方程}

结合上述两个方程,可得到:

\begin{equation}
\ln\left(\frac{I\lambda}{gA}\right) = \ln\left(\frac{hcN_0}{4\pi}\right) - \frac{E_i}{kT}
\end{equation}

实际应用中,常用的形式为:

\begin{equation}
\boxed{\ln\left(\frac{I\lambda}{gA}\right) = -\frac{5040 \times E_i}{T} + C}
\end{equation}

其中$E_i$的单位为eV,$T$的单位为K。

\subsubsection{计算步骤}

\begin{enumerate}
    \item 选择同一元素同一电离态的多条谱线(至少3-4条)
    \item 测量各谱线的积分强度$I$
    \item 查找对应的波长$\lambda$、统计权重$g$、跃迁概率$A$和上能级能量$E_i$
    \item 计算$\ln(I\lambda/gA)$
    \item 以$E_i$为横坐标,$\ln(I\lambda/gA)$为纵坐标作线性拟合
    \item 从直线斜率$m = -5040/T$求得温度$T$
\end{enumerate}

\subsection{双线法}

当只有两条谱线可用时,可使用双线法:

\begin{equation}
\boxed{T = \frac{5040 \times (E_2 - E_1)}{\ln\left[\frac{I_1\lambda_1g_2A_2}{I_2\lambda_2g_1A_1}\right]}}
\end{equation}

\subsection{Saha-Boltzmann图法}

对于同时包含原子线和离子线的情况:

\begin{equation}
\ln\left(\frac{I_{\text{ion}}\lambda_{\text{ion}}}{I_{\text{atom}}\lambda_{\text{atom}}} \times \frac{g_{\text{atom}}A_{\text{atom}}}{g_{\text{ion}}A_{\text{ion}}}\right) = -\frac{5040(E_{\text{ion}} + E_{\text{ionization}} - E_{\text{atom}})}{T} + \ln\left(\frac{2U_{\text{ion}}}{U_{\text{atom}}}\right) - \ln(N_e) + 15.68
\end{equation}

\section{电子密度计算}

\subsection{Stark展宽法}

\subsubsection{理论基础}

在等离子体中,谱线的Stark展宽主要由电子碰撞引起,展宽程度与电子密度成正比。

\subsubsection{基本公式}

完整的Stark展宽公式为:

\begin{equation}
\Delta\lambda_{1/2} = 2w \times \left(\frac{N_e}{10^{16}}\right) \times \left[1 + 1.75A\left(\frac{N_e}{10^{16}}\right)^{1/4} \times (1-0.75R)\right]
\end{equation}

其中:
\begin{itemize}
    \item $\Delta\lambda_{1/2}$:谱线半高全宽(FWHM)
    \item $w$:Stark展宽参数
    \item $N_e$:电子密度(cm$^{-3}$)
    \item $A$:离子展宽参数
    \item $R$:离子-电子密度比
\end{itemize}

\subsubsection{简化公式}

在低密度近似下($N_e < 10^{17}$ cm$^{-3}$):

\begin{equation}
\boxed{N_e = \frac{\Delta\lambda_{1/2}}{2w} \times 10^{16}}
\end{equation}

\subsubsection{常用谱线的Stark展宽参数}

\textbf{氢原子H$_\alpha$线(656.3 nm):}
\begin{equation}
w = 0.548 \text{ Å (at } T = 10000 \text{ K)}
\end{equation}

\begin{equation}
\boxed{N_e = 8.02 \times 10^{12} \times (\Delta\lambda_{1/2})^{1.46} \text{ cm}^{-3}}
\end{equation}

\textbf{氢原子H$_\beta$线(486.1 nm):}
\begin{equation}
w = 0.319 \text{ Å (at } T = 10000 \text{ K)}
\end{equation}

\begin{equation}
\boxed{N_e = 1.26 \times 10^{13} \times (\Delta\lambda_{1/2})^{1.46} \text{ cm}^{-3}}
\end{equation}

\subsection{Saha方程法}

\subsubsection{基本公式}

Saha方程描述了电离平衡:

\begin{equation}
\frac{N_{\text{ion}} \cdot N_e}{N_{\text{atom}}} = \frac{2U_{\text{ion}}}{U_{\text{atom}}} \left(\frac{2\pi m_e kT}{h^2}\right)^{3/2} \exp\left(-\frac{E_{\text{ionization}}}{kT}\right)
\end{equation}

\subsubsection{实用形式}

\begin{equation}
\boxed{\log(N_e) = \log\left(\frac{N_{\text{ion}}}{N_{\text{atom}}}\right) + \log\left(\frac{2U_{\text{ion}}}{U_{\text{atom}}}\right) + 15.68 - \frac{5040 \times E_{\text{ionization}}}{T}}
\end{equation}

其中:
\begin{itemize}
    \item $U_{\text{ion}}, U_{\text{atom}}$:离子和原子的配分函数
    \item $E_{\text{ionization}}$:电离能(eV)
    \item $m_e$:电子质量
\end{itemize}

\subsection{连续辐射法}

连续辐射强度与电子密度的平方成正比:

\begin{equation}
I_{\text{continuum}} = C \times N_e^2 \times \exp\left(-\frac{h\nu}{kT}\right)
\end{equation}

因此:

\begin{equation}
\boxed{N_e = \sqrt{\frac{I_{\text{continuum}} \times \exp(h\nu/kT)}{C}}}
\end{equation}

\section{实际计算示例}

\subsection{温度计算示例}

假设测得Fe原子的几条谱线数据如表\ref{tab:fe_lines}所示:

\begin{table}[h]
\centering
\caption{Fe原子谱线数据}
\label{tab:fe_lines}
\begin{tabular}{@{}ccccc@{}}
\toprule
波长 (Å) & 强度 (a.u.) & $g$ & $A$ (s$^{-1}$) & $E_i$ (eV) \\
\midrule
5269.5 & 1000 & 7 & $6.4 \times 10^7$ & 4.28 \\
5328.0 & 800 & 5 & $5.1 \times 10^7$ & 4.26 \\
5371.5 & 600 & 9 & $7.2 \times 10^7$ & 4.30 \\
\bottomrule
\end{tabular}
\end{table}

\textbf{计算步骤:}

1. 计算$\ln(I\lambda/gA)$值:
   \begin{align}
   \text{Line 1: } &\ln\left(\frac{1000 \times 5269.5}{7 \times 6.4 \times 10^7}\right) = \ln(11.76) = 2.46 \\
   \text{Line 2: } &\ln\left(\frac{800 \times 5328.0}{5 \times 5.1 \times 10^7}\right) = \ln(16.70) = 2.81 \\
   \text{Line 3: } &\ln\left(\frac{600 \times 5371.5}{9 \times 7.2 \times 10^7}\right) = \ln(4.97) = 1.60
   \end{align}

2. 进行线性拟合,从斜率求得温度。

\subsection{电子密度计算示例}

测得H$_\alpha$线的半高全宽为0.8 Å:

\begin{equation}
N_e = 8.02 \times 10^{12} \times (0.8)^{1.46} = 5.8 \times 10^{15} \text{ cm}^{-3}
\end{equation}

\section{注意事项和误差分析}

\subsection{温度计算的注意事项}

\subsubsection{选线原则}
\begin{itemize}
    \item 选择同一元素同一电离态的谱线
    \item 避免自吸收严重的谱线
    \item 上能级能量差应足够大($> 1$ eV)
    \item 谱线应无重叠,强度适中
\end{itemize}

\subsubsection{主要误差来源}
\begin{itemize}
    \item 光谱仪分辨率和波长精度限制
    \item 原子数据($A$值、$g$值)的不确定性
    \item 非热力学平衡效应
    \item 自吸收和自发射效应
    \item 基体效应和光谱干扰
\end{itemize}

\subsection{电子密度计算的注意事项}

\subsubsection{Stark展宽法的限制}
\begin{itemize}
    \item 需要足够高的光谱分辨率($< 0.1$ Å)
    \item 必须扣除仪器展宽的影响
    \item Stark展宽参数的温度依赖性
    \item 适用的电子密度范围:$10^{14} - 10^{18}$ cm$^{-3}$
\end{itemize}

\subsubsection{仪器展宽修正}

实际的Stark展宽需要扣除仪器展宽:

\begin{equation}
\boxed{\Delta\lambda_{\text{Stark}} = \sqrt{\Delta\lambda_{\text{measured}}^2 - \Delta\lambda_{\text{instrument}}^2}}
\end{equation}

\subsubsection{温度修正}

Stark展宽参数的温度依赖性:

\begin{equation}
\boxed{w(T) = w_{\text{ref}} \times \left(\frac{T}{T_{\text{ref}}}\right)^\alpha}
\end{equation}

其中$\alpha$通常为0.5-1.0。

\section{实验条件对测量的影响}

\subsection{激光参数的影响}

\begin{itemize}
    \item \textbf{激光能量}:影响等离子体温度和密度
    \item \textbf{脉冲宽度}:影响等离子体的形成和演化
    \item \textbf{聚焦条件}:影响功率密度和等离子体特性
\end{itemize}

\subsection{检测参数的影响}

\begin{itemize}
    \item \textbf{延迟时间}:影响等离子体的演化阶段
    \item \textbf{门宽}:影响信号积分和时间分辨率
    \item \textbf{光谱分辨率}:影响谱线展宽的准确测量
\end{itemize}

\section{结论}

LIBS等离子体的温度和电子密度是表征等离子体状态的重要参数。Boltzmann图法是测量温度的标准方法,而Stark展宽法是测量电子密度的主要手段。在实际应用中,需要根据具体的实验条件和要求选择合适的计算方法,并注意各种误差来源的影响。

准确的等离子体诊断不仅有助于理解LIBS的物理机制,还为优化实验条件、提高分析精度提供了重要依据。随着LIBS技术在工业应用中的推广,等离子体诊断技术的重要性将日益凸显。

\section*{参考文献}

\begin{enumerate}
    \item Cremers, D. A., \& Radziemski, L. J. (2013). \textit{Handbook of laser-induced breakdown spectroscopy}. John Wiley \& Sons.
    \item Miziolek, A. W., Palleschi, V., \& Schechter, I. (Eds.). (2006). \textit{Laser-induced breakdown spectroscopy (LIBS): fundamentals and applications}. Cambridge University Press.
    \item Griem, H. R. (1997). \textit{Principles of plasma spectroscopy}. Cambridge University Press.
    \item NIST Atomic Spectra Database. \url{https://www.nist.gov/pml/atomic-spectra-database}
\end{enumerate}

\end{document} 